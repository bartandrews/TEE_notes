%\documentclass[twocolumn,preprintnumbers,amsmath,amssymb]{revtex4-1}
%\documentclass[aps,prl,reprint,amsmath,amssymb]{revtex4-1}
\documentclass[floatfix,showpacs,amsmath,amsfonts,amssymb,aps,twocolumn, prb,groupedaddress]{revtex4-1}

\usepackage[table,x11names]{xcolor}
\usepackage{graphicx}% Include figure files
\usepackage{dcolumn}% Align table columns on decimal point
\usepackage{amsthm}
\usepackage{bm}% bold math
\usepackage{siunitx}
\usepackage{nicefrac}
\usepackage{todonotes}
\usepackage{mathtools}
\usepackage{float}
\usepackage{braket}
\usepackage{makecell} 
\usepackage{multirow}
\usepackage{braket}

\definecolor{darkred}{rgb}{0.6,0.,0.}
\definecolor{darkgreen}{rgb}{0.,0.5,0.}
\definecolor{darkblue}{rgb}{0.,0.,0.6}
\usepackage[ breaklinks, colorlinks=true, linkcolor=darkred, citecolor=darkgreen, urlcolor=darkblue]{hyperref}

\newcommand{\inputgnuplottex}[1]{\input{#1}}
\newcommand{\includegraphicsgood}{\includegraphics}

\usepackage[inactive,tightpage]{preview}
%\PreviewBorder=12pt\relax
\PreviewBorder=0pt
\PreviewMacro[*[[!]{\inputgnuplottex}
\PreviewMacro[*[[!]{\includegraphicsgood}
%\renewcommand \PreviewBbAdjust{-12pt -\PreviewBorder -12pt \PreviewBorder}

\newcommand{\figref}[1]{Fig.~\ref{fig:#1}}
\newcommand{\secref}[1]{Section~\ref{sec:#1}}
\newcommand{\eqnref}[1]{Eq.~(\ref{eq:#1})}
\newcommand{\tabref}[1]{Table~\ref{#1}}
\newcommand\papercomment[1]{\textcolor{teal}{***#1}}
\newcommand{\AuGe}{$\mathrm{Au}_x\mathrm{Ge}_{1-x}$\,}
\newcommand{\AuGex}[2]{$\mathrm{Au}_{#1}\mathrm{Ge}_{#2}$\,}
\newcommand{\rhoxx}{\rho_{\mathrm{xx}}}
\newcommand{\rhoxy}{\rho_{\mathrm{xy}}}
\newcommand{\dd}[1]{\ensuremath{\operatorname{d}\!{#1}}}
\newcommand{\ddd}[2]{\ensuremath{\operatorname{d}^{#2}\!{#1}}}
\newcommand{\pdb}[2]{\frac{\partial #1}{\partial #2}}
\DeclareRobustCommand{\orderof}{\ensuremath{\mathcal{O}}}
\newcommand{\dBbydt}{\nicefrac{\dd{B}}{\dd{t}}}
\newcommand{\Beff}{\textbf{B}^{\mathrm{eff}}}
\newcommand{\kF}{k_\mathrm{F}}
\newcommand{\kr}{k_\mathrm{r}}
\newcommand{\krF}{k_\mathrm{r}^\mathrm{F}}
\newcommand{\kz}{k_\mathrm{z}}
\newcommand{\epsr}{\epsilon_\mathrm{r}}
\newcommand{\epsk}{\epsilon_\mathrm{k}}
\newcommand{\epsf}{\epsilon_\mathrm{F}}
\newcommand{\ek}{\epsilon_\mathrm{k}}
\newcommand{\ekr}{\epsilon_{\kr}}
\newcommand{\ekrF}{\epsilon_{\krF}}
\newcommand{\ekz}{\epsilon_{\kz}}
\newcommand{\eL}{\epsilon_{\lambda}}
\newcommand{\eF}{\epsilon_\textrm{F}}
\newcommand{\wwF}{\sqrt{b^2 + 2\ekrF\eL}}
\newcommand{\vB}{\mathbf{B}}
\newcommand{\vD}{\mathbf{\Delta}}
\newcommand{\nn}{\hat{\mathbf{n}}}
\newcommand{\sigmatot}{\sigma^{\mathrm{tot}}}
\newcommand{\punc}[1]{\,#1}
\newcommand{\diffd}{\text{d}}

\newcommand{\e}[1]{\text{e}^{#1}}
\newcommand{\cmplxi}{\text{i}}
\newcommand{\dr}{\,\mathrm{d}\mathbf{r}}
\renewcommand{\vec}[1]{\mathbf{#1}}
\newcommand{\neweqnline}{\nonumber\\}
\newcommand{\vecgrk}[1]{\boldsymbol{#1}}
%\newcommand{\ket}[1]{|#1\rangle}

\graphicspath{{./figures/}}

\pagenumbering{gobble}

\begin{document}

\onecolumngrid

\begin{center}
	\textbf{\large Effective Abelian order from non-Abelian fractional quantum Hall states in the Hofstadter model}
\end{center}
\vspace{-1em}
\tableofcontents
\vspace{1em}

This is a test.
This is another test.

The single-particle Hamiltonian for the Hofstadter model in this project is given as
\begin{equation}
\label{eq:single_ham}
H_0 = \sum_{\braket{ij}} \left[ t e^{\mathrm{i}\theta_{ij}} a_i^\dagger a_j + \text{H.c.} \right],
\end{equation}
where $t$ is the hopping amplitude, $e^{\mathrm{i}\theta_{ij}}$ is the Peierls phase factor, $a^\dagger(a)$ are generic creation(annihilation) operators, and $\braket{ij}$ denotes nearest-neighbor hoppings on a square lattice. The total Hamiltonian is given as $H=H_0+H_I$, where $H_I$ is the interaction Hamiltonian. Since the kinetic energy in a given Landau level is quenched for the fractional quantum Hall effect, the physics is governed by $H_I$.

In each case below, we compute the topological entanglement entropy from an extrapolation of the area law of entanglement to the thermodynamic limit.

\section{Investigation of Abelian states}

For Abelian states, the topological entanglement entropy is given as $\gamma=\ln(\sqrt{s})$ at fillings $\nu=r/s$.~\cite{Estienne15}

\subsection{Laughlin states}

\subsubsection{Hardcore bosons at $\nu=1/2$}

\begin{equation}
H_I = U \sum_i \rho_i \rho_i,
\end{equation}
where $\rho=a^\dagger a$ is the density operator and $U=\infty$.

\subsubsection{NN fermions at $\nu=1/3$}

\begin{equation}
H_I = V \sum_{\braket{ij}} \rho_i \rho_j,
\end{equation}
where $V=10$.

We can compare this graph directly with Fig. 3 in the paper by Schoonderwoerd et al.~\cite{Schoonder19}.

\subsection{Hierarchy states}

\subsubsection{Hardcore bosons at $\nu=2/3$}

\begin{equation}
H_I = U \sum_i \rho_i \rho_i,
\end{equation}
where $\rho=a^\dagger a$ is the density operator and $U=\infty$.

\subsubsection{NN fermions at $\nu=2/5$}

\begin{equation}
H_I = V \sum_{\braket{ij}} \rho_i \rho_j,
\end{equation}
where $V=10$.

\section{Investigation of non-Abelian states}

\subsection{Unitary states}

For Moore-Read states, the topological entanglement entropy is given as $\gamma=\ln (2\sqrt{s})$ at fillings $\nu=1/s$, where $s$ is odd for bosons and even for fermions.~\cite{Glasser15}

\subsubsection{Bosonic Moore-Read state at $\nu=1$}

\begin{equation}
H_I = \sum_{n=2}^N \frac{U_n}{n!} \sum_{i} (b^\dagger_i)^n (b_i)^n,
\end{equation}
where $N$ is the total (even) number of particles, $U_2=0$ and $U_{n>2}=\infty$. Note that $:\prod_{l=1}^n \rho_i:=(b^\dagger_i)^n (b_i)^n$. This is equivalent to the three-body hardcore condition $(b^\dagger_i)^3=0$ and $(b_i)^3=0$.~\cite{Zhu15}

We can compare these results with studies on the Haldane honeycomb lattice model.~\cite{Zhu15, Wang12}  The crucial point is that Haldane pseudopotential is a constant for this state (see appendix A of the paper by Liu et al.~\cite{Liu12}).

\subsubsection{Fermionic Moore-Read state at $\nu=1/2$}

\begin{equation}
H_I = \sum_{n=2}^N \frac{U_n}{n!} \sum_{i} (c^\dagger_i)^n (c_i)^n,
\end{equation}
where $U_2=0$, $U_3=10$ and $U_{n>3}=\infty$.

\subsection{Non-unitary states}

\subsubsection{Bosonic Gaffnian state at $\nu=2/3$}

\subsubsection{Fermionic Gaffnian state at $\nu=2/5$}

For the fermionic Gaffnian state, the topological entanglement entropy is known to be $\gamma=1.44768$.~\cite{Estienne15}

We need to construct a toy Hamiltonian whose ground state is the Gaffnian wavefunction.

A construction of the Gaffnian parent Hamiltonian for the cylinder geometery is outlined in the paper by Lee et~al.~\cite{Lee15}

An explicit form of the bosonic Gaffnian Hamiltonian in terms of derivatives of delta functions is given in the paper by Papic in Eq.~A4.~\cite{Papic14}

The original paper to detail the Gaffnian is by Simon et al.~\cite{Simon07}



\section{Tuning from effective Abelian to non-Abelian order}

For contested states, such as $\nu=2/5$, we expect to find the Abelian value for the topological entanglement entropy, $\gamma=\ln(\sqrt{5})$, with short-range interactions. After increasing the range of the interactions, we should notice an increase of the entanglement entropy towards the non-Abelian value of $\gamma=1.44768$.

\bibliographystyle{apsrev4-1}
\bibliography{TEE_notes}

\end{document}

